\documentclass[a4paper,11pt]{article}
\usepackage[UTF8]{ctexcap}
%\usepackage{xeCJK}
%\setCJKmainfont{SimSun}

\usepackage{geometry}
 \geometry{
     a4paper,
     total={210mm,297mm},
     left=23mm,
     right=23mm,
     top=23mm,
     bottom=23mm,
 }
\linespread{1.2}

\usepackage{fancyvrb, graphicx, hyperref}
\usepackage{listings}

\usepackage[final]{showkeys}
\usepackage{hyperref}
\usepackage{amsmath}

\usepackage{float, graphicx}

\usepackage{indentfirst}
\setlength{\parindent}{2em}

\usepackage{fancyhdr}
\pagestyle{fancy} \lhead{高级编程技术}\rhead{\thepage}
\setcounter{tocdepth}{3}

%\usepackage[chinese]{babel} % English language/hyphenation


\usepackage{listings}
\usepackage{color}
\definecolor{mygreen}{rgb}{0,0.6,0}
\definecolor{mygray}{rgb}{0.5,0.5,0.5}
\definecolor{mymauve}{rgb}{0.58,0,0.82}
\lstset{ %
  backgroundcolor=\color{white},   % choose the background color; you must add \usepackage{color} or \usepackage{xcolor}
  basicstyle=\footnotesize,        % the size of the fonts that are used for the code
  breakatwhitespace=false,         % sets if automatic breaks should only happen at whitespace
  breaklines=true,                 % sets automatic line breaking
  captionpos=b,                    % sets the caption-position to bottom
  commentstyle=\color{red},    % comment style
  deletekeywords={...},            % if you want to delete keywords from the given language
  escapeinside={\%*}{*)},          % if you want to add LaTeX within your code
  extendedchars=true,              % lets you use non-ASCII characters; for 8-bits encodings only, does not work with UTF-8
  frame=single,                    % adds a frame around the code
  keepspaces=true,                 % keeps spaces in text, useful for keeping indentation of code (possibly needs columns=flexible)
  keywordstyle=\color{blue},       % keyword style
  otherkeywords={*,...},            % if you want to add more keywords to the set
  numbers=left,                    % where to put the line-numbers; possible values are (none, left, right)
  numbersep=5pt,                   % how far the line-numbers are from the code
  numberstyle=\tiny\color{mygray}, % the style that is used for the line-numbers
  rulecolor=\color{black},         % if not set, the frame-color may be changed on line-breaks within not-black text (e.g. comments (green here))
  showspaces=false,                % show spaces everywhere adding particular underscores; it overrides 'showstringspaces'
  showstringspaces=false,          % underline spaces within strings only
  showtabs=false,                  % show tabs within strings adding particular underscores
  stepnumber=2,                    % the step between two line-numbers. If it's 1, each line will be numbered
  stringstyle=\color{mymauve},     % string literal style
  tabsize=2,                       % sets default tabsize to 2 spaces
  title=\lstname                   % show the filename of files included with \lstinputlisting; also try caption instead of title
}


\title{高级编程技术}
\author{计科一班~赖少凡 13349047}
\date{January 8, 2016}


\begin{document}

\begin{titlepage}
	\newcommand{\HRule}{\rule{\linewidth}{0.5mm}}
	\center
	
	\includegraphics[width=0.3\textwidth]{./sysu.jpg}~\\[1cm]
	\textsc{\LARGE 中山大学}\\[1.5cm]
	\textsc{\Large 高级编程技术}\\[0.5cm]
%	\textsc{\large Minor Heading}\\[0.5cm]
	
	\HRule \\[0.8cm]
	{ \huge \bfseries 轻量级在线聊天框架——Awoo}\\[0.4cm]
	\HRule \\[1.5cm]
	
	\begin{minipage}{0.4\textwidth}
		\begin{flushleft} \large
			\emph{作者:}\\
			赖少凡
		\end{flushleft}
	\end{minipage}
	~
	\begin{minipage}{0.4\textwidth}
		\begin{flushright} \large
		\emph{指导老师:} \\
			李才伟
		\end{flushright}
	\end{minipage}\\[4cm]


	{\large 2016年1月8日}\\[3cm]

	\end{titlepage}

\tableofcontents

\section*{摘要}
    我实现了一个开源轻量级的在线聊天框架——Awoo,该框架分为服务器端和Windows客户端两个部分。用户可以在任意网络上部署自己的服务器端,然后在联网情况下使用Windows客户端的在线聊天等功能。该框架架构清晰,且采用统一的RESTful API,二次开发者能够很轻松地将该框架拓展到网页、平板、不同的操作系统或平台上。此外,我还为试用该框架的用户搭建了一个免费的服务器端(\cite{1})。

\section{框架介绍}
    \subsection{框架名字来源}
        Awoo为狼吼叫的拟声词。在自然界中,狼吼是狼群交流的一个重要途径\textsuperscript{\cite{2}},如聚集其他狼群狩猎等。狼吼在风速较小的时候能够传达到很远的地方。本框架取名于这个意思,旨在用最简单、最轻量级的方法,将信息在网络上传播。
    \subsection{服务器端}
        与传统的qq、微信等聊天软件不同,Awoo的服务器端采用的是\textbf{可分布部署}的模式,用户可以将Awoo Server部署在自己想要的任何地方:互联网上、局域网上、甚至本机上。如一个公司内网需要一个聊天软件,但是又不想与外网相连,这时候就可以将Awoo Server部署在局域网的服务器上。同时我也部署了一个免费的、全球范围可访问的服务器端(\cite{1}),连接到互联网的用户可以直接使用\cite{1}进行聊天。\par
        服务器端使用Flask\textsuperscript{\cite{3}}框架进行编写。Flask是一个轻量级的网站后端框架,其学习代价小、灵活性高,适合开发轻量级的服务。
    \subsection{Windows客户端}
        Awoo提供了Windows平台下的客户端,需要在.Net 4.5环境中运行。该客户端调用了Awoo提供的所有基础功能,如登陆、注册、在线聊天、绘图聊天等。此外,该客户端还提供一些本地的个性化选项,如记忆登陆名、自定义字体等。\par
        Awoo for Windows是利用WPF技术开发的,使用C\#语言编写。
    \subsection{框架接口}
        由于用户端的平台(如平板、手机、PC等)差异巨大,且就PC 而言也有许多不同的系统(如Windows、Linux、MacOS等),因此在时间和精力的限制下,我只开发了Windows平台的客户端。但是,在接口的设计上,为了契合多平台多系统的特性,Awoo 特地采用了RESTful API来作为接口形式。RESTful广泛用于浏览器与网站服务器的交互(如简单的GET和POST请求),同时许多开源、商用服务平台也是采用了API接口。\par
        使用RESTful能将Awoo框架的可移植性、二次开发功能发挥到极致。二次开发者可以在网页、平板、手机、各种系统的PC上开发自己的聊天软件。而且服务器端

\section{框架细节}
    \subsection{服务器端}
        由于服务器端不是本次报告的重点,因此会忽略一些Python细节,着重介绍服务器端与客户端如何交互。
        \subsubsection{API}
            Awoo共有20余个API,均是以RESTful(主要是POST)的形式进行交互,格式均为Application/Json。在用户需要获取一些数据时(如获取信息),客户端通过发送一个指定内容的包(如用户名、对方用户名、token)到服务器的指定接口(如/api/msg/fetch),便可收到一个指定格式的回复。通过对回复内容进行解析、处理,最后呈现给用户。
        \subsubsection{验证方式}
            首先,用户在成功登陆后会产生一个临时生成的token(类似于网页中的cookie)。在用户登陆后需要操作时,将通过用户名加token的方式验证,无需再次验证密码,减少风险。在新的版本,Awoo的token将有一定的生命周期,以增加安全性;在两个token交接的短时间内同时承认两个token的合法性,之后改用新的token,这样客户端就会有一定的时间执行token的交接,此方式受微信公众号验证方式的启发\textsuperscript{\cite{4}}。
        \subsubsection{存储}
            Awoo支持离线消息,即用户在下线时依旧可以接受信息,在重新登录时未读取信息将会被推送到客户端。由于目前使用人数较少,因此不会定时对数据库进行备份和清空,用户可以查看从注册到任何时刻的所有聊天记录。

    \subsection{Windows客户端}
        本节将结合着使用界面以及一些核心代码进行介绍:
        \subsubsection{公共代码}
            由于程序中许多代码段是共享的,因此我在Common.cs中定义了一个Shared类,它是一个静态类,里面存储了一些“全局”参数,同时实现了一些常用的代码片段(如移动窗体、图片转化、信息发送接收等)。其余窗体类通过调用Shared里的函数来实现代码复用。此外,客户端用于和服务器端交互的数据包格式均定义在Common.cs中,作为一个类存在。
        \subsubsection{信息交互}
            在客户端与服务器端交互时,大致经历了以下步骤:\par
            \begin{enumerate}
                \item 从窗体获取数据(如用户名、密码)
                \item 将数据打包成一个对象(如new FormLogin(username, password))
                \item 使用RestSharp\textsuperscript{\cite{6}}将对象序列化(如"\{username='admin',password='123456'\}")
                \item 将序列化的结果发送给服务器
                \item 从服务器接受到反馈(否则抛出异常)
                \item 将反馈解序列化成一个对象(如Reply(result='succeed'))
                \item 从对象中读取信息,并做相应处理
            \end{enumerate}
            其中发送和接受到的信息要被(解)序列化成为两个对象,因此需要定义两个类。对于不同的交互(如登陆、发送消息、更新个人信息等),所用的表单格式也不一样,即每个交互都需要两个不同的类来进行抽象(有些交互可以共用一个类)。这样一来需要编写多个发送、接收、处理函数,我编写了一个泛型函数来完成这件事情:
            \begin{figure}[H]
                \centering
                \includegraphics[scale=1.0]{ss14.png}
                \caption{发送、接收信息的函数模板,其中T1为发送的类,T2为接收的类,"where T2:new()"保证了T2不能是一个抽象类,而是一个能够实例化的类}
            \end{figure}
        \subsubsection{登陆界面}
            一打开Awoo,可以见到一个登陆界面。\par
            \begin{figure}[H]
                \centering
                \includegraphics[scale=1.0]{ss1.png}
                \caption{登陆界面}
            \end{figure}
            为了美观和新颖,程序中所有界面没有使用默认的Windows 窗体模板,而是通过自己设计的WPF模板来做到边框圆角、高光反射以及磨砂效果。其嵌套结构如下:\par
            \begin{itemize}
                \item Window:最外层为窗口本身,因为需要去掉windows的模式,因此需要将WindowStyle设为None(取消顶部按钮以及标题栏)、令ResizeMode="NoResize"(去掉四边的拉伸边框)、令
                    AllowsTransparency="True"(隐去默认窗体)
                \item Border:为自定义边框层,通过设置CornerRadius属性来调整圆角的平滑度,同时将BroderBrush设置为白色,凸显出反光效果。
                \item Grid:为内容层,里面放置着主内容。其中有一个自定义的按钮作为新模板的关闭键,有一个label作为新模板的标题栏。
            \end{itemize}
            此外,整个窗体(除了按钮、文本框等交互控件),其他部分都是可以通过按紧鼠标来拖拽窗口。其原理是通过user32.dll文件中的sendmessage函数,对窗体的hanlde发送一个特定的信息,使得窗体能够被拖拽。代码大致如下:\par
            \begin{figure}[H]
                \centering
                \includegraphics[scale=1.0]{ss2.png}
                \caption{移动窗口代码,通过发送Windows消息来实现}
            \end{figure}
            软件将会把上一次登陆的服务器地址、用户名、密码记录在一个本地文件中。在用户打开软件时尝试加载,在用户登陆时更新。如果没有注册账户,也可以点击注册按钮进行注册。注册界面如下:\par
            \begin{figure}[H]
                \centering
                \includegraphics[scale=1.0]{ss3.png}
                \caption{注册窗口}
            \end{figure}
            在成功登陆之后,软件会弹出一个对话框,提示用户已登录。
        \subsubsection{主界面}
            在成功登陆之后,可能需要等待3~5秒(视网速而定),这时客户端将从服务器端加载一些必要的信息,并且准备进入主界面。详情见图\par
            \begin{figure}[H]
                \centering
                \includegraphics[scale=1.0]{ss5.png}
                \caption{主界面,从上到下分别为:用户信息、朋友列表、搜索/添加朋友按钮、个人设置。在朋友列表中,右键可以对一个好友发起聊天,或删除该好友。}
            \end{figure}
        \subsubsection{聊天界面}
            在主界面右键朋友、选择send message to ... 便可进入聊天窗口,或直左键接双击朋友所在的行也可以。\par
            \begin{figure}[H]
                \centering
                \includegraphics[scale=1.0]{ss6.png}
                \caption{聊天窗口,从上到下分别为:朋友名字、聊天内容、工具栏、输入栏、操作栏}
            \end{figure}
            其中工具栏共有三个小按钮,分别为:发送图片、绘制图片并发送、查看历史消息。操作栏的按钮功能为:强制刷新信息(程序默认会0.3秒获取信息并刷新)、关闭聊天窗口和发送消息(ctrl+enter为发送快捷键)。\par
            点击发送图片按钮后,将能从文件系统中选取图片,并发送给好友。此处使用到了Windows自带的文件查看窗口。图片以Base64格式编码成字符串发送,接受后又重新解码成图片:\par
            \begin{figure}[H]
                \centering
                \includegraphics[scale=1.0]{ss7.png}
                \caption{发送图片功能}
            \end{figure}
            \begin{figure}[H]
                \centering
                \includegraphics[scale=1.0]{ss8.png}
                \caption{对图像进行Base64编码解码的过程}
            \end{figure}
            点击绘制图片并发送按钮后,会弹出一个绘图框。绘图框支持铅笔画图、直线画图、方框画图、椭圆画图等,其功能是在李才伟老师\textsuperscript{\cite{5}}的WPF绘图程序上改进得到的。在绘制完图片后,WPF通过截屏将窗体的画布部分截取出来,然后编码成Base64发送,其原理与直接发送图片类似。\par
            \begin{figure}[H]
                \centering
                \includegraphics[scale=1.0]{ss9.png}
                \caption{发送文字、图片、绘图}
            \end{figure}
        \subsubsection{接受信息}
            在上述操作后,test账户接收到了admin的消息,并且这些消息是未读的,因此admin的头像将在test账户的朋友列表中闪动,提示test去读取消息:\par
             \begin{figure}[H]
                \centering
                \includegraphics[scale=1.0]{ss10.png}
                \caption{未读消息闪烁提示}
            \end{figure}
            \begin{figure}[H]
                \centering
                \includegraphics[scale=1.0]{ss11.png}
                \caption{test收到admin发送的信息}
            \end{figure}
            历史信息功能与接收信息类似,只不过客户端将获取所有admin与test的历史信息。
        \subsubsection{个性化设置}
            \begin{figure}[H]
                \centering
                \includegraphics[scale=1.0]{ss12.png}
                \caption{主窗口->settings->个性化设置,从上到下依次是:聊天窗口字体设置,个人介绍,密码修改,头像修改}
            \end{figure}        
            每个用户可以个性化设置自己聊天窗口的字体,这些信息将保存在本地,下次在本机使用相同用户登陆时就无需再次设置。\par
            在个性化设置窗口还可以填写个人介绍、更改密码、更改头像。更改头像时如果选择的头像超过64*64像素,程序将自动进行压缩。
        \subsubsection{搜索、添加、删除好友}            
            在主窗口下的search按键可以搜索名字中带有给定字符的用户名,搜索到后可以通过Add按钮添加对方为好友。此时系统会自动发送一句问候给对方。如果想删除一个联系人,只需要在好友列表中右键->detele即可删除该好友。\par
             \begin{figure}[H]
                \centering
                \includegraphics[scale=1.0]{ss13.png}
                \caption{搜索、添加、删除好友}
            \end{figure}
        \subsubsection{自动隐藏窗口}
            当主界面被拖动到屏幕顶端时,会自动缩成一个窄长条。当鼠标再次划过时,会恢复原界面。
            \begin{figure}[H]
                \centering
                \includegraphics[scale=1.0]{ss15.png}
                \caption{主界面在顶部的自动隐藏}
            \end{figure}
            

    \subsection{其他细节}
        \subsubsection{测试账户}
            测试账户名为test,密码为123456。
        \subsubsection{库}
            主要使用的库为Python的Flask\textsuperscript{\cite{3}}(服务器后端),以及C\#的RestSharp\textsuperscript{\cite{6}}(网络通信)。
        \subsubsection{版本控制}
            开发过程中通过Git进行版本控制,历史版本以及源码可以在\cite{7}中获取。服务器端的文件在"\\server"中,客户端的文件在"\\local\\Awoo"中。
        \subsubsection{延时}
            因为本程序使用的是单线程通信,因此在网络延时高的环境下,软件界面可能因为网络延时而会有卡顿现象。
        \subsubsection{字体对话框}
            Windows自带的FontDialog对WPF的支持并不是十分好,因此有一定可能在更改字体时导致字体无法转换而出错。
            
\section{总结}
    在这次大作业中,我觉得我的作业(Awoo)主要有以下特点:\par
    \begin{itemize}
        \item 架构导向而非单一软件导向:与单纯开发一个一对一聊天软件不同,我的目的是开发一个完整的、从服务器端到客户端的框架。
        \item 框架灵活、可二次开发、易拓展:与传统的聊天软件不同,Awoo相对灵活,可以部署在不同的网络范围内。我的框架还是开源的,代码结构较为清晰,方便二次开发。同时我采用了RESTful而非socket等方式进行通信,使得Awoo的可拓展性更好(如网页端对socket支持差)。
        \item 轻量级:Awoo以轻量级为特性,客户端大小只有500KB。虽然某些设计牺牲掉了一部分效率,但是这个框架做到了”麻雀虽小,五脏俱全“的效果。
        \item WPF:Awoo客户端基于.Net Framework 4.5.2,并且所有的窗体都是通过WPF实现的,WPF相比MFC是比较现代的软件框架。而且我对Awoo的设计也做了数次修改,使其外观更加好看、时尚。
        \item 编程技巧:在Awoo开发过程中,我运用到了C\#的各种特性,如:泛型函数、泛型约束、静态类(以实现全局变量)等。
    \end{itemize}
    在开发Awoo的过程中,我也遇到了不少难题:\par
    \begin{itemize}
        \item 虽然Visual Studio是一个十分强大的武器,但是学会高效实用VS并不是一朝一夕的事情。在开发中,我在摸索中学到了定义跳转、代码自动生成、自动修复错误等小功能,这些功能能够使开发更加快捷高效。其中有一点我觉得VS做的十分出色,我还掌握了对程序包管理控制台(PM)的运用。这个包管理器能够有效地下载、安装、卸载、管理第三方包,当工程中需要一个第三方插件时,只需要一条命令就能部署完整个环境,而不用手动去下载、编译、配置相对路径。
        \item XAML对窗体的设计并没有想象中那么简单,有人将XAML与HTML进行类比,但是其区别还是很多的。在HTML中,我们习惯用CSS来编写样式;而在XAML中,除了Style之外,还有诸如Resources、Template等诸多概念,它们在App的生成中扮演着不同的角色。同时,抛开语言的区别,WPF也与Delphi相比也增加了很多新特性(属性、行为、概念),如何利用这些特性来实现一个功能并不是那么简单,往往需要查询大量的文档、例子才能学习到其精髓。
        \item 网络编程本身并不是一个难点,有第三方的库可以实现自动的发送和接收,难点在于对数据在本地的处理。因为C\#不像python一样,它是一个静态类型的语言。对于不同的表单和不同的回复需要设计不同的类来抽象其数据结构,同时又需要统一收发网络包的接口,这就需要对C\#的泛型编程有一定的了解。
    \end{itemize}
    而Awoo也存在很多可以改进的地方:如更多的个性化选项、多线程异步获取信息、对所以异常的正常处理等等。总而言之,在这次大作业设计中,我学到了许多前沿的C\#、.Net、WPF开发知识,并且挑战了自己对一个大框架的设计、对UI界面的设计,是相当有收获的一次作业。

\clearpage
\renewcommand\refname{参考文献}
\begin{thebibliography}{99}
    \bibitem{1} \url{http://awoo.hapd.info/} 免费的、全球互联网范围内的服务器端
    \bibitem{2} \url{http://www.wolf.org/wolf-info/basic-wolf-info/biology-and-behavior/communication/} How wolf communicate, Access Date Jan 8, 2016
    \bibitem{3} \url{http://flask.pocoo.org/} Flask框架
    \bibitem{4} \url{http://mp.weixin.qq.com/wiki/14/9f9c82c1af308e3b14ba9b973f99a8ba.html} 微信公众号对token的处理 
    \bibitem{5} \url{http://sist.sysu.edu.cn/~isscwli/} 李才伟老师主页
    \bibitem{6} \url{http://restsharp.org/} RestSharp主页
    \bibitem{6} \url{https://github.com/piscesdream/awoo} Awoo工程主页
\end{thebibliography}

\end{document}
